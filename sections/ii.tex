\section{Gruppen}

\subsection{Magmen}

\defi{Magma} $(M,*)$ ist eine Menge mit einer Verknüpfung $*:M\times M \to M$
\begin{itemize}
    \item ist die Verknüpfung assoziativ, so nennt man das Magma Halbgruppe
    \item Eine Halbgruppe mit neutralem Element heißt Monoid
    \item es gibt auch kommutative Magmen
    \item Bsp: Abb(D,D) mit Komposition ist ein Magma. Leeres Magma. Triviales Magma.
    \item $U\subset M$ heißt Untermagma wenn $U*U\subset U$. Untermagmen kann man schneiden und diese bleiben Untermagmen
    \item Sei $X\subset M$, das Magmenerzeugnis $\langle X\rangle$ von $X$ ist der Schnitt alles Untermagmen, die $X$ enthalten.
    \item für assoz. Magmen gilt: $\langle X\rangle = \cup_{n\in\mathbb{N}}X_n$ mit $X_1=X, X_{n+1}=X*X_n$ 
\end{itemize}

\subsection{Der Gruppenbegriff}

\defi{Gruppe} 
\begin{itemize}
    \item Sei $(M,*)$ ein Monoid mit neutralem Element $e$. Ein Element $x$ heißt invertierbar, wenn ein $y \epsilon M$ exisitiert, sodass $x*y=y*x=e$. Bezeichne mit $M^x$ die Menge aller invertierbaren Elemente in M.
    \item Eine Gruppe ist ein Monoid, in dem jedes Element invertierbar ist.
    \item Sei $(M,*)$ eine Gruppe. $(M,*)$ ist kommutativ/abelsch wenn sie als Magma kommutativ ist.
\end{itemize}

\defi{Untergruppe}
Sei $(G,*)$ eine Gruppe. Dann ist eine Untergruppe von G ein nichtleeres Untermagma U, das unter der Inversenbildung abgeschlossen ist. \\
Schreibe: $U \leq \ G$ 
\begin{itemize}
    \item Die Gruppe $(\mathbb{Z},+)$ ist eine Untergruppe von $(\mathbb{Q},+)$.
    \item Die Gruppe der ganzen Zahlen $\mathbb{Z}$ mit der Addition als Verknüpfung. Die triviale Untergruppe ist die ${0}$. AUßerdem gilt für jede natürliche Zahl n die Teilmenge \\
    $n \mathbb{Z}:={nk \vert k \epsilon \mathbb{Z}}$ \\
    Wir halten fest: Die Untergruppen von $\mathbb{Z}$ sind genau die Mengen $n \mathbb{Z}$ mit $n \epsilon \mathbb{N}_{0}$
\end{itemize}

\bem{} Durchschnitt von Untergruppen: Sei G Gruppe, I nichtleere Menge und für jedes $i \epsilon I$ gibt es eine Untergruppe $U_i$ von G geben. Dann ist $\cap_{i \epsilon I} U_i$ eine Untergruppe von G.

\defi{Gruppenerzeugnis}: Sei M Teilmenge der Gruppe G, sei I die Menge aller Untergruppen von G, die M enthalten. \\
$\langle M \rangle:=\cap_{i \epsilon I} i$ 
ist Gruppenerzeugnis von M oder die von M erzeugte Untergruppe von G.
\begin{itemize}
    \item Es ist die kleinste Untergruppe von G, die M enthält. 
\end{itemize}

\defi{zyklisch}: Eine Gruppe G heißt zyklisch, wenn es ein Element $a\epsilon G$, sodass $G=\langle{a}\rangle$

\defi{Ordnung} Die Mächtigkeit/Kardinalität einer Gruppe nennt man auch Ordnung. Die Ordnung eines Elemtens $g\epsilon G$ ist definiert als die Ordnung der von g erzeugten Untergruppe.
\begin{itemize}
    \item Wenn $g\epsilon G$ endliche Ordnung, dann ist diese gleich der kleinsten natürlichen Zahl k, für die $g_k=e_G$ gilt.
\end{itemize}

\bem{} Satz von Lagrange: Sei G endl. Gruppe und H Untergruppe von G. Dann gilt: $\#H\mid\#G$

\defi{Index}: Wenn $H \leq \ G$ zwei Gruppen, dann heißt die Anzahl der Äquivalenzklassen der Index von H in G. Schreibe $(G:H)$. \\
Es gilt demnach für alle Gruppen $(G:H)=\frac{\#G}{\#H}$.

\subsection{Homomorphismen zwischen Gruppen}
\defi{Gruppenhomomorphismus}: Sei $(G,*)$ und $(H,\cdot)$ zwei Gruppen. Ein Homomorphismus von G nach H ist eine Abbildung $f:G\rightarrow H$, für die gilt:

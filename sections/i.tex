\section{Teilbarkeit}

\defi{} $d$ teilt $n$ $\Leftrightarrow$ $\exists t: t \cdot d = n$ \\
Begriffe: Teiler, Vielfaches, ggT, kgV

\bem{ggT} $ggT(0,m)=m$, $ggT(m,n) = ggT(m,n-m)$

\defi{Euklidischer Algorithmus.} $a\stackrel{!}{<}b$. Solange $b-a>0$: $b:= b-a$

\bem{} mit dem eukl. Alg. kann man auch den ggT als Linearkombination darstellen.

\defi{Kongruenz} $k\equiv l (\mod n)$ $\Leftrightarrow n | k-l$

\bem{}
\begin{itemize}
    \item $g=ggT(n,m)$ $ \Rightarrow ggT(\frac{n}{g},\frac{m}{g}) = 1$
    \item $m,n$ teilerfremd und $m \mid nu$ $\Leftrightarrow m\mid u$
    \item $kgV(m,n) \cdot ggT(m,n) = m \cdot n$
\end{itemize}

\defi{Primzahlen} ...

\bem{Fundamentalsatz der Arithmetik:} Jede Zahl hat eine Primfaktorzerlegung

\defi{$p$-adische Bewertung.} $v_p(k)$ ist die höchste Potenz von $p$, die $k$ teilt: $p^{v_p(k)} \mid k$

\bem{}
\begin{itemize}
    \item $b \mid a \gdw \forall p \in \mathbb{P}: v_p(b)\leq v_p(a)$
    \item ggT von $a$ und $b$ ist $\prod_{p\in\mathbb{P}}p^{\min(v_p(a),v_p(b)}$
    \item kgV von $a$ und $b$ ist $\prod_{p\in\mathbb{P}}p^{\max(v_p(a),v_p(b)}$
\end{itemize}

\satz{kl. von Fermat:} $p\in\mathbb{P}$ und $c\in\mathbb{Z}$ $\Rightarrow p \mid c^p-c$ { \footnotesize (hiermit kann prim wiederlegt werden)}

\bem{Lücken:} $k\in\mathbb{N}$. Zwischen $M:=(k+2)!+2$ und $M+k$ liegt keine Primzahl


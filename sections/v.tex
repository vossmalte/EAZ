\section{Teilbarkeit und Primelemente}
\subsection{}
\defi{Assoziert} sind $a$ und $b$ wenn $\exists e \in R^\times: b=a\cdot e$

\bem{Ordnung} entsteht durch Teilbarkeit: $aR^\times \leq bR^\times \gdw a|b$
\bem{teilerfremd} sind $a$ und $b$, wenn die einzigen Teiler Einheiten sind.

\defi{Hauptideal} $I$ ist ein Ideal, für das gilt: $I=Rg=(g)$

\defi{Hauptidealring} ist ein nullteilerfreier kommutativer Ring, in dem jedes Ideal Hauptideal ist.

\bem{ggT:} für $I=Rg={ax+by \mid x,y \in R}$ ist $g=\ggt(a,b)$

\kor{Erzeuger} $Rg=Rh \gdw g,h$ sind assoziert. 

\defi{Euklidischer Ring} ist ein nullteilerfreier kommutativer Ring $R$ mit $\gamma:R\to \NN_0$, wobei $\gamma(r)=0 \gdw r=0$ und $\forall a,b \neq 0 \exists c: \gamma(a-bc)<\gamma(b)$

% weiter bei 5.1.10
\bem{Euklidische Ringe} sind Hauptidealringe

\bem{Chinesischer Restsatz:} $R/(Rrs) \cong R/(Rr) \times R/(Rs)$

\subsection{Arithmetik in Hauptidealringen}

\defi{$m$ irreduzibel:} $\gdw \forall a,b: m=ab \Rightarrow a \text{ oder } b \in R^\times$
\defi{$p$ prim:} $\gdw \forall a,b: p \mid ab \Rightarrow p|a \text{ oder } p \mid b$

\kor{Primelemente} von 0 verschieden sind immer irreduzibel
\kor{Im Hauptidealring} sind irreduzible Elemente prim.

\satz{Primzerlegung} $\mathbb{P}_R$ ist Vertretersystem der Primelemente. Dann ist $\forall r\in R$ ein eindeutiges Produkt assoziert aus Elementen in $\mathbb{P}_R$.

\bem{Restklassenkörper} $R/Rg$ ist Körper $\gdw g$ irreduzibel

\defi{Primideal:} $xy\in I \Rightarrow x\in I \text{ oder } y \in I$





